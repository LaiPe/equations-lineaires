\documentclass[10pt,a4paper,french]{article}
\author{par Léo Peyronnet}
\title{Équations Linéaires}
\date{Octobre 2022}

\usepackage[utf8]{inputenc}
\usepackage[T1]{fontenc}

\usepackage{babel}
\usepackage{listings}
\usepackage{amsfonts}
\usepackage{amsmath}
\usepackage{amssymb}
\usepackage{amsthm}
\usepackage{mathtools}
\usepackage{tabto}
\usepackage{color}
\usepackage[table]{xcolor}

\newenvironment{decompo}{\par\medskip\noindent \textbf{Décomposition:}}{\medskip}
\newenvironment{reccu}{\par\medskip\noindent\textbf{Récurrence:}}{\medskip}
\newenvironment{conv}{\par\medskip\noindent\textbf{Convergence:}}{\medskip}

\definecolor{ao(english)}{rgb}{0.0, 0.5, 0.0}

\lstset{
  aboveskip=3mm,
  belowskip=-2mm,
  basicstyle=\footnotesize,
  breakatwhitespace=false,
  breaklines=true,
  captionpos=b,
  commentstyle=\color{ao(english)},
  deletekeywords={...},
  escapeinside={\%*}{*)},
  extendedchars=true,
  framexleftmargin=16pt,
  framextopmargin=3pt,
  framexbottommargin=6pt,
  frame=tb,
  keepspaces=true,
  keywordstyle=\color{blue},
  language=C,
  literate=
  {²}{{\textsuperscript{2}}}1
  {⁴}{{\textsuperscript{4}}}1
  {⁶}{{\textsuperscript{6}}}1
  {⁸}{{\textsuperscript{8}}}1
  {€}{{\euro{}}}1
  {é}{{\'e}}1
  {è}{{\`{e}}}1
  {ê}{{\^{e}}}1
  {ë}{{\¨{e}}}1
  {É}{{\'{E}}}1
  {Ê}{{\^{E}}}1
  {û}{{\^{u}}}1
  {ù}{{\`{u}}}1
  {â}{{\^{a}}}1
  {à}{{\`{a}}}1
  {á}{{\'{a}}}1
  {ã}{{\~{a}}}1
  {Á}{{\'{A}}}1
  {Â}{{\^{A}}}1
  {Ã}{{\~{A}}}1
  {ç}{{\c{c}}}1
  {Ç}{{\c{C}}}1
  {õ}{{\~{o}}}1
  {ó}{{\'{o}}}1
  {ô}{{\^{o}}}1
  {Õ}{{\~{O}}}1
  {Ó}{{\'{O}}}1
  {Ô}{{\^{O}}}1
  {î}{{\^{i}}}1
  {Î}{{\^{I}}}1
  {í}{{\'{i}}}1
  {Í}{{\~{Í}}}1,
  morekeywords={*,...},
  numbers=left,
  numbersep=10pt,
  numberstyle=\tiny\color{black},
  rulecolor=\color{black},
  showspaces=false,
  showstringspaces=false,
  showtabs=false,
  stepnumber=1,
  stringstyle=\color{gray},
  tabsize=4,
  title=\lstname,
}


\begin{document}
\maketitle
Compte rendu du TP consistant à programmer et comparer certaines méthodes de résolution de systèmes linéaires.
\section{Rappel des méthodes}
\subsection{Méthode de Gauss (directe)}
Dans le système $Ax = b$, la matrice inversible $A \in M_{n,n}(\mathbb{R})$ s'exprime sous la forme $A = M.N$ avec les matrices $M$ et $N$ où $M$ est facilement inversible et $N$ est triangulaire. Ainsi, $Ax = b \Leftrightarrow M.Nx = b \Leftrightarrow Nx = M^{-1}b$.
La méthode de Gauss consiste en deux étapes:
\begin{itemize}
\item La triangulation: où l'on cherche $y \in \mathbb{R}^n$ tel que $My=b \Leftrightarrow y=M^{-1}b$.
\item La résolution "facile": où l'on cherche $x \in \mathbb{R}^n$ tel que $Nx=b \Leftrightarrow x=N^{-1}y$.  
\end{itemize}
\subsection{Méthodes itératives}
Toujours dans le système $Ax = b$, la matrice inversible $A \in M_{n,n}(\mathbb{R})$ peut également s'exprimer sous la forme $A=M-N$ avec $M$ facilement inversible. Ainsi, $Ax = b \Leftrightarrow (M-N)x = b \Leftrightarrow Mx-Nx = b \Leftrightarrow Mx = Nx + b \Leftrightarrow x = M^{-1}(Nx + b)$.
 
Soit $F(x)=M^{-1}(Nx + b)$ tel que $F(x)=x$ pour la solution de $Ax=b$. $x$ est donc un point fixe de la fonction $F(x)$. Si $F(x)$ est une application strictement contractante, alors la suite ci-dessous converge vers un point fixe de F(x).
$$
\left\{
	\begin{array}{ll}
		x_0 \in \mathbb{R}^n \\
		x_{n+1}=F(x_n)
	\end{array}
\right.
$$
$$
\Leftrightarrow\left\{
	\begin{array}{ll}
		x_0 \in \mathbb{R}^n \\
		x_{n+1}=M^{-1}(Nx_n+b)
	\end{array}
\right.
$$
\subsubsection{Méthode Jacobi}
\begin{decompo}
$A=D-E-F$ avec:
\begin{itemize}
\item $D$ la diagonale de $A$.
\item $-E$ la partie sous la diagonale de $A$.
\item $-F$ la partie sur la diagonale de $A$.
\end{itemize}
Ainsi, $M=D$ ; $N=E+F$ ; $A=M-N$.
\end{decompo}

\begin{reccu}
$$
\left\{
	\begin{array}{ll}
		x_0 \in \mathbb{R}^n \\
		x_{n+1}=D^{-1}((E+F)x_n+b))
	\end{array}
\right.
$$

\end{reccu}

\begin{conv}
Si la matrice $A$ a une diagonale strictement dominante, alors $F(x)=M^{-1}(Nx+b)$ est strictement contractante et $x_n$ tend vers la solution. 
\end{conv}

\subsubsection{Méthode Gauss-Seidel}
\begin{decompo}
Avec la même décomposition,\\
$M=D-E$ ; $N=F$ ; $A=M-N$.
\end{decompo}

\begin{reccu}
$$
\left\{
	\begin{array}{ll}
		x_0 \in \mathbb{R}^n \\
		x_{n+1}=(D-E)^{-1}(Fx_n+b))
	\end{array}
\right.
$$

\end{reccu}

\begin{conv}
Si la matrice $A$ a une diagonale strictement dominante ou si A est symétrique définie positive, alors $F(x)=M^{-1}(Nx+b)$ est strictement contractante et $x_n$ tend vers la solution.  
\end{conv}
\section{Présentation des programmes}
\subsection{Algorithme de Gauss simple et résolution de matrice triangulaire supérieure}

\begin{lstlisting}
void gauss(float ** A,float * B,int taille){
    for (int k=0;k<taille-1;k++){
        for (int i=k+1;i<taille;i++){
            float piv=A[i][k]/A[k][k];
            for (int j=k;j<taille;j++){
                A[i][j]-=piv*A[k][j];
            }
            B[i]-=piv*B[k];
        }
    }
}

float * ResTrigSup(float ** A,float * B,int taille){
    float * X=declTab(taille);
    int n=taille-1;

    X[n]=B[n]/A[n][n];
    for (int i=n-1;i>=0;i--){
        float somm=0;
        for (int j=i+1;j<=n;j++){
            somm+=A[i][j]*X[j];
        }
        X[i]=(1/A[i][i])*(B[i]-somm);
    }
    return X;
}
\end{lstlisting}

\subsection{Méthode Jacobi}
\begin{lstlisting}
float E(float ** A,float * X,float * B,int taille){
    float norme=0;
    float v=0;
    for (int i=0;i<taille;i++){
        //Produit Vectoriel A*X
        float AXi=0; //Somme du produit A[i]*X[i]
        for (int j=0;j<taille;j++){
            AXi+=A[i][j]*X[i];
        }
        v=AXi-B[i]; // v représente le produit vectoriel AX de la i-ème ligne auquel on soustrait B[i]
        norme+=puiss(v,2); //somme des v de toutes les lignes ("taille" lignes) au carré
    }
    return sqrtf(norme); // <=> ||AX-B||
}

float * jacobi(float e,float ** A,float * B,int taille){
    //initialisation de X (x0)
    float * X=declTab(taille);
    for (int y=0;y<taille;y++){
        X[y]=0; //vecteur nul
    }
    //algo jacobi
    //RAPPEL: Ax=b <=> A=D-E-F
    int k=1;
    while (E(A,X,B,taille)>=(1/e)){
        for (int i=0;i<taille;i++){
            //Produit Vectoriel -((E+F)*x) pour la i-ème ligne
            float somm=0;
            for (int j=0;j<taille;j++){
                if (j!=i){
                    somm+=A[i][j]*X[j];
                }
            }
            //
            X[i]=((1/A[i][i])*(B[i]-somm)); // <=> D-1 * (B-(-(E+F)*x))
        }
        k++;
    }
    printf("k=%d\n",k); //affichage du nombre d'itérations
    return X;
}
\end{lstlisting}
\subsection{Méthode Gauss-Seidel}
\begin{lstlisting}
// La fonction E est sous-entendue

    //initialisation de X (x1)
    float * X0=declTab(taille);
    float * X1=declTab(taille);
    for (int y=0;y<taille;y++){
        X1[y]=0; //vecteur nul
        X0[y]=0;
    }
    //algo gauss_seidel
    int k=1;
    while (E(A,X1,B,taille)>=(1/e)){
        for (int i=0;i<taille;i++){
            float somm=0;
            for (int j=0;j<i;j++){
                somm+=A[i][j]*X1[j];
            }
            for (int j=i+1;j<taille;j++){
                somm+=A[i][j]*X0[j];
            }
            //
            X0[i]=X1[i];
            X1[i]=((1/A[i][i])*(B[i]-somm));
            
        }
        k++;
    }
    printf("k=%d\n",k); //affichage du nombre d'itérations
    return X1;
}
\end{lstlisting}
\section{Tables et graphiques sur les jeux d'essais}
\section{Conclusion}
\end{document}